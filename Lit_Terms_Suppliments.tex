\documentclass{article}
\usepackage[utf8]{inputenc}

\title{Terminology and Supplements}
\date{December 2019}
\usepackage{float}
\usepackage{natbib}
\usepackage{graphicx}
\usepackage{todonotes}
\usepackage{etoolbox}
\usepackage{csquotes}

\begin{document}
\maketitle
\section{Literature review}
\subsection{Parametric and unit-selective Speech Synthesis \cite{zen2009statistical}}
According this paper:
\blockquote[\cite{zen2009statistical}]{
Statistical parametric synthesis might be most simply described as generating the average of some sets of similarly sounding speech  segments.    This  contrasts  directly  with  the  target  in unit-selection synthesis that retains natural unmodified speech units}
Each method has its own advantages. The quality gap of synthesized speech between these methods is domain dependent.\cite{zen2009statistical}

\cite{esling2018generative}

\subsection{Future tasks and other thoughts}
\subsubsection{Data}
Could drum data retrieval become its own project/paper? Using the field of automatic drum transcription, cutting loops up and other relevant fields. Nsynth \cite{nsynth2017} database doesn't have many percussion sounds as the emphasis is on pitches and notes. 
\section{Terminology and Measurements}

\bibliographystyle{plain}
\bibliography{references}
\end{document}

